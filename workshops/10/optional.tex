\def \firstA {
    \item Найдите уравнения проекций линии пересечения эллипсоида $3x^2 + 4y^2 + 5z^2 = 36$ и сферы $x^2 + y^2 + z^2 = 9$ на координатные плоскости. Что представляет собой сечение?
}

\def \firstB {
    	\item Найдите уравнения проекций линии пересечения эллипсоидов \\ $x^2 + 2y^2 + 3z^2 = 4$, $3x^2 + 5y^2 + 6z^2 = 10$ на координатные плоскости. Что представляет собой сечение?
}

\def \firstC {
    \item Найдите уравнения проекций линии пересечения поверхностей $x^2 + y^2 - z^2 = 1$, $x^2 - y^2 = 2z$ на координатные плоскости. Что представляет собой сечение?
}
\def \secondA {
    	\item Найдите центр сечения эллипсоида $x^2 + 2y^2 + 4z^2 = 40$ плоскостью \\ $x + y + 2z = 5$
}

\def \secondB {
    	\item Найдите центр сечения эллипсоида $x^2 + 2y^2 + 4z^2 = 40$ плоскостью $x + y + z = 7$
}

\def \secondC {
    	\item Найдите центр сечения гиперболоида $x^2 + 2y^2 - 4z^2 = -4$ плоскостью \\ $x + y + 2z = 2$
}

\def \third {
   \item Две прямолинейные образующие гиперболоида вращения $x^2 + y^2 - z^2 = 1$ пересекаются в точке, принадлежащей плоскости $z = h$. Найдите угол между ними
}

\def \fA {
    \item Найти множество точек поверхности $x^2 + y^2 - z^2 = 1$, в которых пересекаются её взаимно ортогональные прямолинейные образующие
}

\def \fB {
    \item Найти множество точек поверхности $x^2 - y^2 = 2z$, в которых пересекаются её взаимно ортогональные прямолинейные образующие
}

\def \fC {
    \item Найти множество точек поверхности $x^2 - 4y^2 = 4z$, в которых пересекаются её взаимно ортогональные прямолинейные образующие
}
 





