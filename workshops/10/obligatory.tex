\begin{enumerate}
    \item Запишите уравнения, задающие данное преобразование плоскости и определите, является ли оно афинным:
        \begin{enumerate}
            \item центральная симметрия относительно точки $M(x_0, y_0)$;
            \item сжатие к прямой $2x - y + 5 = 0$ с коэффициентом $\alpha = 2$;
            \item преобразование, переводящее точки $A(-2, 0)$, $B(2, -1)$ и $C(0, 4)$ \\в $A'(-2, 1)$, $B'(2, 1)$, и $C'(0, 1)$ соответственно.
        \end{enumerate}
   		
   \item  Найдите все неподвижные точки афинного преобразования, заданного формулами ${x' = 2x - 3y +3}, {y' = -2x + 2y - 6}$.
	
	\item Найдите все инвариантные прямые афинного преобразования, заданного формулами ${x' = 5x + 3y +1}, {y' = -3x - y}$.
	
	\item В параллелограмм $ABCD$ вписан эллипс. Точки касания $M, N$ и $K$ лежат на сторонах $AB$, $BC$ и $CD$ соответственно. Найдите площадь треугольника $NBM$, если площадь треугольника $CNK$ равна 20, а $NC = 2BN$.
	
	\item На какой угол нужно повернуть прямую $3x - 4y + 25 = 0$ вокруг точки $M(-7, 1)$, чтобы её образ был параллелен оси абсцисс?
	
	\item Дано афинное преобразование, заданное формулами ${x' = 4x-3y+1}$,  ${y'=3x+4y+5}$. На прямой ${x+y+2=0}$ найдите точку, которая переходит в точку, также принадлежащую этой прямой.
\end{enumerate}