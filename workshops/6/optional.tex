\def \point {
    \item Решить систему векторных уравнений в пространстве: $(\vec{x}, \vec{a}) = p$, $(\vec{x}, \vec{b}) = q$, $(\vec{x}, \vec{c}) = s$ (векторы $\vec{a}$, $\vec{b}$ и $\vec{c}$ некомпланарны)
}
\def \parametr {
    \item Найдите, при каких значениях $\alpha$ прямая $x = \frac{y}{\alpha} = 2 - z $
          \begin{enumerate}
	        \item пересекает плоскость $3\alpha^2 x + \alpha y + z - 4a = 0$
	        \item параллельна этой плоскости 
	        \item лежит в этой плоскости 
	    \end{enumerate}
}

\def \plane {
    \item Плоскость $\alpha$ перпендикулярна прямой, проходящей через точки $A(3, 5, 1)$ и $B(5, 1, 3)$. Составить общее уравнение плоскости $\alpha$, если расстояние $\rho$ от неё до точки $M(1, 2, 3)$ равно 5. Система координат декартова прямоугольная.  
}

\def \toward {
    	\item В ПДСК задано уравнение плоскости $Ax + By + Cz + D = 0$. Составьте уравнение плоскости, параллельной данной и находящейся в 2 раза ближе к началу координат.
}

\def \rotation {
    \item Составить уравнение плоскостей, проходящих через прямую $ \frac{x-1}{3} =  \frac{y-1}{5} =  \frac{z + 2}{4}$ и равноудаленных от точек $A(1, 2, 5)$ и $B(3, 0, -1)$
}


