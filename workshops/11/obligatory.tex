
\subsubsection*{Новая тема:}

\begin{enumerate}
    \item Запишите формулы, задающие произведение $fg$ данных афинных преобразований $f$ и $g$, и охарактеризовать это произведение геометрически (СК --- общая декартова).
    
    f --- гомотетия с центром в точке $M(2, -1)$ и коэффициентом $\alpha=1/2$;
    
    g --- центральная симметрия относительно точки $N(3, 1)$.
    \item Найдите координаты векторов, задающих главные направления данного афинного преобразования:
    \begin{tasks}(2)
    	\task $\begin{cases}x' = x-y\\y'=x+y\end{cases}$
    	\task $\begin{cases}x' = -4x+7y\\y'=8x+y\end{cases}$
    \end{tasks}
    
    \item Представьте афинное преобразование в виде произведения $f=h_2h_1g$, где $g$ --- ортогональное преобразование, а $h_1$ и $h_2$ --- сжатия к двум взаимно перпендикулярным прямым. 
    $$\begin{cases}x' = -4x+7y\\y'=8x+y\end{cases}$$
    
    \item Образующие цилиндра параллельны вектору $\vec a(1, 1, 1)$, его направляющая --- окружность $x^2 + y^2 = 2z$, $x^2+y^2+x^2=8$. Напишите уравнение цилиндра.
    \item Найдите уравнение конуса с вершиной в точке $M(1, 1, 1)$, касающегося сферы $x^2 + y^2 + z^2 = 2$.
\end{enumerate}

\subsubsection*{На повторение:}
\begin{enumerate}
    \item Найдите расстояние между прямыми $x+y+z-1=0, x-y+z+1=0$ и $x=0$, заданными в ПДСК.
    \item В правом ОНБ три вектора $\vec a(1, 2, 3), \vec b(4, 5, 6)$  и $\vec c(2, 3, 1)$ образуют треугольную призму. Найти объем призмы и высоту, проведенную к основанию (за основание считать треугольник, образованный векторами $\vec a$ и $\vec b$). 
 \end{enumerate}