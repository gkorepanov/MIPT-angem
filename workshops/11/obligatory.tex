
\subsubsection*{Новая тема:}

\begin{enumerate}
    \item Запишите формулы, задающие произведение $fg$ данных афинных преобразований $f$ и $g$, и охарактеризовать это произведение геометрически (СК --- общая декартова).
    
    f --- гомотетия с центром в точке $M(2, -1)$ и коэффициентом $\alpha=1/2$;
    
    g --- центральная симметрия относительно точки $N(3, 1)$.
    \item Найдите координаты векторов, задающих главные направления данного афинного преобразования:
    \begin{tasks}(2)
    	\task $\begin{cases}x' = x-y\\y'=x+y\end{cases}$
    	\task $\begin{cases}x' = -4x+7y\\y'=8x+y\end{cases}$
    \end{tasks}
    
    \item Представьте афинное преобразование в виде произведения $f=h_2h_1g$, где $g$ --- ортогональное преобразование, а $h_1$ и $h_2$ --- сжатия к двум взаимно перпендикулярным прямым. 
    $$\begin{cases}x' = -4x+7y\\y'=8x+y\end{cases}$$
    
    \item Найдите уравнение прямого кругового конуса с вершиной в начале координат и направлением оси, определяемым вектором $a(1, 1, 1)$, зная, что образующие конуса составляют с его осью угол $\arccos\left(\frac{1}{\sqrt{3}}\right)$
    
    \item Найти уравнение прямого кругового цилиндра, проходящего через точку $M(1, 1, 2)$, и имеющего ось $x = 1 + t$, $y = 2 + t$, $z = 3 + t$
    
\end{enumerate}

\subsubsection*{На повторение:}
\begin{enumerate}
    \item Найдите расстояние между прямыми $x+y+z-1=0, x-y+z+1=0$ и $x=0$, заданными в ПДСК.
    \item В правом ОНБ три вектора $\vec a(1, 2, 3), \vec b(4, 5, 6)$  и $\vec c(2, 3, 1)$ образуют треугольную призму. Найти объем призмы и высоту, проведенную к основанию (за основание считать треугольник, образованный векторами $\vec a$ и $\vec b$). 
 \end{enumerate}