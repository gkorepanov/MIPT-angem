\begin{enumerate}
	
	\item В правом ОНБ на векторах $\vec a (0, -2, 2), \vec b (-2, -2, 3), \vec c(-4, 3, 1)$ построен тетраэдр. Найдите объем тетраэдра и его высоту, проведенную к основанию (за основание считайте треугольник, построенный на векторах $\vec a$ и $\vec b$).

    \item Точка $M$ определяется радиус-вектором $\mathbf{r_0}$. Запишите уравнение:
        \begin{enumerate}
	        \item прямой, проходящей через точку $M$ перпендикулярно плоскости $(\mathbf{r}, \mathbf{n})=D$
	        \item плоскости, проходящей через точку M перпендикулярно прямой $[\mathbf{r}, \mathbf{a}]=\mathbf{b}$
	    \end{enumerate}

	\item Найдите необходимое и достаточное условие, при котором плоскости ${(\mathbf{r}, \mathbf{n_1}) = D_1}$ и ${(\mathbf{r}, \mathbf{n_2}) = D_2}$:
	    \begin{enumerate}
	        \item пересекаются
	        \item параллельны, но не совпадают
	        \item совпадают
	    \end{enumerate}
	 
	\item Найдите необходимое и достаточное условие, при котором плоскость $(\mathbf{r}, \mathbf{n}) = D$ и прямая $\mathbf{r}=\mathbf{r_0} + \mathbf{a}t$:
	    \begin{enumerate}
	        \item имеют единственную общую точку
	        \item не имеют общих точек
	        \item имеют бесконечное число общих точек
	    \end{enumerate}
	

	\item Составить уравнение прямой, пересекающей прямую $\vec{r} = \vec{r_1} + \vec{a} t$ под прямым углом и проходящей через точку $M_0(\vec{r_0})$,  не лежащую на данной прямой (перпендикуляра, опущенного из точки на прямую).
	\item Найти расстояние между параллельными прямыми $[\vec{r}, \vec a] = \vec{b_1}$ и $[\vec{r}, \vec a] = \vec{b_2}$.
	

	\item Найти уравнение прямой, пересекающей скрещивающиеся прямые $\mathbf{r} = \mathbf{r_1} + \mathbf{a_1}t$ и $\mathbf{r} = \mathbf{r_2} + \mathbf{a_2}t$ под прямым углом (общий перпендикуляр).
   
   
\end{enumerate}