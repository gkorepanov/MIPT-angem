\def \geomsense {
   \item Даны неколлинеарные вектора $\vec a$ и $\vec b$ и число $p\in\mathbb{R}$. Решить векторное уравнение и объяснить геометрический смысл всех его решений, а также частного решения, ортогонального $\vec a$ и $\vec b$:
    $$ (\vec x, \vec a, \vec b)= p $$ 
 }
 
%\def \first {
%	\item В пространстве даны две прямоугольные системы координат $O, \vec e_1, \vec e_2, \vec e_3$ $O^\prime, \vec e_1^\prime, \vec e_2^\prime, \vec e_3^\prime$. Точки $O$ и $O^\prime$ различны, а концы векторов $\vec e_i$ и $\vec e_i^\prime$, отложенных соответственно из $O$ и $O^\prime$ ($i=1,2,3$), совпадают. Найти координаты точки пространства в первой системе координат, если известны её координаты $x^\prime, y^\prime, z^\prime$ во второй системе координат.
%}

\def \naprcos {
	\item В пространстве даны две прямоугольные системы координат $\{O, \vec e_1, \vec e_2, \vec e_3\}$ и $\{O', \vec e_1', \vec e_2', \vec e_3'\}$. Начало второй системы координат имеет в первой системе координаты $(-1, 3, 5)$. Вектор $\vec e_1'$ образует углы, равные $60^\circ$ с векторами $\vec e_1$ и $\vec e_2$ и острый угол с $\vec e_3$. Вектор $\vec e_2'$ компланарен с векторами $\vec e_1$ и $\vec e_2$ и образует с вектором $\vec e_2$ острый угол. Тройки   $\{O, \vec e_1, \vec e_2, \vec e_3\}$ и $\{O', \vec e_1', \vec e_2', \vec e_3'\}$ ориентированы одинаково. Найдите координаты точки пространства в первой СК, если известны ее координаты $(x', y', z')$ во второй СК.
}

\def \second {
	\item Координаты $(x, y, z)$ каждой точки пространства в первой системе координат выражаются через координаты $(x', y', z')$ этой же точки во второй системе координат соотношениями:
	\begin{flalign*}
		& x = a_{11} x' + a_{12} y' + a_{13} z' + \beta_1 \\
		& y = a_{21} x' + a_{22} y' + a_{23} z' + \beta_2 \\
		& z = a_{31} x' + a_{32} y' + a_{33} z' + \beta_3
	\end{flalign*}
	
	\begin{enumerate}
		\item Первая СК -- прямоугольная. Найдите критерий прямоугольности второй~СК.
		\item Найдите необходимое и достаточное условие одинаковой ориентации базисов первой и второй~СК.
	\end{enumerate}
}

\def \vectask {
 	\item В трапеции $ABCD$ диагонали пересекаются в точке $E$, а длины оснований соотносятся как $DC/AD = 2/3$. Найти координаты точки $M$ в системе координат $\{E, \overrightarrow{EA}, \overrightarrow{EB}\}$, если известны её координаты $(1, 1)$ в системе $\{A, \overrightarrow{AB}, \overrightarrow{AD}\}$. 
 }
 
 \def \system {
 	\item Решите систему векторных уравнений в пространстве:
 	$$(\vec a, \vec x) = p;\quad (\vec b, \vec x) = q;\quad (\vec c, \vec x) = s,$$
 	если векторы $\vec a, \vec b, \vec c$ некомпланарны.
 }
 
%\def \gram {
%	\item Докажите, что для любых векторов $\vec x, \vec y, \vec z, \vec a, \vec b, \vec c$ верно тождество
%	$$ (\vec a, \vec b, \vec c) (\vec x, \vec y, \vec z) =
%	\begin{vmatrix}
%	(\vec a, \vec x) & (\vec b, \vec x) & (\vec c, \vec x) \\
%	(\vec a, \vec y) & (\vec b, \vec y) & (\vec c, \vec y) \\
%	(\vec a, \vec z) & (\vec b, \vec z) & (\vec c, \vec z)
%	\end{vmatrix}
%$$}

\def \ubiy {
   \item В системе координат с базисом $\{\vec e_1, \vec e_2, \vec e_3 \}$ даны точки $A(2, 1, -1)$, $B(3, 0, 2)$, $C(5, 1, 1)$ и $D(0, -1, 3)$, являющиеся вершинами тетраэдра. Найдите объём тетраэдра, если $|\vec e_1|=|\vec e_2|=|\vec e_3|= 1, \angle (\vec e_1, \vec e_2)=  \angle (\vec e_1, \vec e_3)=  \angle (\vec e_2, \vec e_3) = 60^\circ$
 }