\begin{enumerate}
	\item Дайте определение смешанного произведения и опишите его геометрический смысл:\ \hrulefill\par\hrulefill
    
    \item В базисе $\{\vec e_1, \vec e_2, \vec e_3\}$ заданы вектора $\vec a = \alpha_1 \vec e_1 + \alpha_2 \vec e_2 + \alpha_3 \vec e_3$, $\vec b = \beta_1 \vec e_1 + \beta_2 \vec e_2 + \beta_3 \vec e_3$ и~$\vec c = \gamma_1 \vec e_1 + \gamma_2 \vec e_2 + \gamma_3 \vec e_3$.
  
    \begin{enumerate}[topsep=0.5cm, itemsep=1cm]
        \item $(\vec a, \vec b, \vec c)= $
        \item В случае ОНБ $(\vec a, \vec b, \vec c)= $
    \end{enumerate}
    
    \item Для произвольных $\vec a$, $\vec b$ и $\vec c$ выберите верные равенства:
    \begin{tasks}(1)
        \task $(\vec a, \vec b, \vec c)= (\vec c, \vec a, \vec b)$
        \task $(\vec a, \vec b, \vec c)= -(\vec c, \vec a, \vec b)$ 
        \task $(\alpha \vec a_1 + \beta \vec a_2, \vec b, \vec c)= \alpha(\vec a_1, \vec b, \vec c) + \beta (\vec a_2, \vec b, \vec c)$
        \task $(\vec a, \alpha \vec b_1 + \beta \vec b_2,  \vec c)= -\alpha(\vec a, \vec b_1, \vec c) - \beta (\vec a, \vec b_2, \vec c)$
    \end{tasks}
    
    \item Заданы системы координат $\{O, \vec e_1, \vec e_2, \vec e_3\}$~(1) и $\{O', \vec e_1^\prime, \vec e_2^\prime, \vec e_3^\prime\}$~(2). Начало второй системы~$O'$ имеет в первой координаты $(1, 1, -1)$.  Базисные векторы системы~(2) выражаются через через базисные векторы системы~(1)~как 
     \begin{equation*}
      \begin{cases}
        \vec e_1^\prime =  \vec e_1 +  \vec e_2 +  \vec e_3 \\
        \vec e_2^\prime =  \vec e_1 +   \vec e_3 \\
        \vec e_3^\prime =  \vec e_1 + 2 \vec e_2 + 3 \vec e_3\\
     \end{cases}
    \end{equation*}
    \begin{enumerate}[topsep=0.5cm, itemsep=1cm]
        \item Матрица перехода из (1) в (2) $S = $
        \item Известны координаты точки $M(1, 1, 1)$ в системе координат (2). Координаты~в~(1)~равны:\vspace{1cm}  
     \end{enumerate}
     
   \item Записать критерий компланарности векторов $\vec a, \vec b, \vec c$ в терминах смешанного произведения: \hrulefill\par\hrulefill

\end{enumerate}