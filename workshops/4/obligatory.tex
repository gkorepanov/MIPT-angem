\begin{enumerate}	
	\item Вычислите площадь треугольника, построенного на векторах $(1\ 2)$ и  $(-1\ 3)$, заданных в базисе $\{\vec e_1, \vec e_2\}$. Известно, что $|\vec e_1| = 2$, $|\vec e_2| = 3$, $\angle (\vec e_1, \vec e_2) = 60^\circ$.
	\item В правом ОНБ три вектора $\vec a(2, 1, -3),\ \vec b(2, 1, -2)$  и $\vec c(5, -2, 0)$ образуют треугольную призму. Найти объем призмы и высоту, проведенную к основанию (за основание считать треугольник, образованный векторами $\vec a$ и $\vec b$).

	\item Даны системы координат $\{O, \vec e_1, \vec e_2, \vec e_3\}$ и $\{O^\prime, \vec e_1^\prime, \vec e_2^\prime, \vec e_3^\prime\}$, причем
	\begin{flalign*}
		& \vec e_1^\prime = \alpha_{11} \vec e_1 + \alpha_{12} \vec e_2 + \alpha_{13} \vec e_3 \\
		& \vec e_2^\prime = \alpha_{21} \vec e_1 + \alpha_{22} \vec e_2 + \alpha_{23} \vec e_3 \\
		& \vec e_3^\prime = \alpha_{31} \vec e_1 + \alpha_{32}\vec e_2 + \alpha_{33} \vec e_3 \\
		& \overrightarrow{OO^\prime} = \beta_1 \vec e_1 + \beta_2 \vec e_2 + \beta_3 \vec e_3
	\end{flalign*}

	Пусть даны координаты некоторой точки $M(x^\prime, y^\prime, z^\prime)$. Докажите, что
	\begin{flalign*}
		& x = \alpha_{11} x^\prime + \alpha_{21} y^\prime + \alpha_{31} z^\prime + \beta_1 \\
		& y = \alpha_{12} x^\prime + \alpha_{22} y^\prime + \alpha_{32} z^\prime + \beta_2 \\
		& z = \alpha_{13} x^\prime + \alpha_{23} y^\prime + \alpha_{33} z^\prime + \beta_3
	\end{flalign*}

   
   \item
   Задана система координат $\{O, \vec e_1, \vec e_2\}$ (1), а СК $\{O', \vec e_1^\prime, \vec e_2^\prime\}$ (2) получена из первой поворотом векторов $\vec e_1$ и $\vec e_2$ на угол $60^\circ$. Начало второй системы имеет в первой координаты $(1, 1)$. Найти матрицу перехода из (1)~в~(2), если $\{\vec e_1, \vec e_2\}$ -- ОНБ. 
    
    \item В треугольнике $ABC$ на стороне $AB$ отмечена точка $K$, а на стороне $AC$ -- точка~$L$. Известно, что $AK : KB = 1 : 2$, а $AL : LC = 1 : 5$. Точка~$P$ --точка пересечения медиан $\triangle ABC$, а точка $M$ имеет координаты $(2, -4)$ в СК $\{C, \overrightarrow{BL}, \overrightarrow{AP}\}$. Найдите координаты точки $M$ в СК $\{B, \overrightarrow{AK}, \overrightarrow{AL}\}$.


   
   \item    
   В правильной шестиугольной пирамиде $SABCDEF$ c вершиной $S$ точка $M$ является серединой основания. Найти координаты точки пространства в системе координат $\{A, \vec{AB}, \vec{AF}, \vec{AS}\}$, если известны её координаты $x^\prime, y^\prime, z^\prime$ в системе координат $\{S, \vec{SC}, \vec{SD}, \vec{SM}\}$.
    
    \item Докажите, что если векторы $\vec e_1, \vec e_2, \vec e_3$ некомпланарны, т.е. образуют базис в пространстве, то их попарные векторные произведения $[\vec e_1, \vec e_2], [\vec e_3, \vec e_1], [\vec e_1, \vec e_2]$ также образуют базис. Такой базис называется \textbf{биортогональным} к базису $\{\vec e_1, \vec e_2, \vec e_3\}$ и обладает рядом важных свойств, используемых в математике.
   
   
\end{enumerate}