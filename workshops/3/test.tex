\begin{enumerate}
	\item Дайте определение:
	    \begin{enumerate}
	    	\item Cкалярное произведение $\vec a$ и $\vec b$:\ \hrulefill\par\hrulefill
	    	\item Векторное произведение $\vec a$ и $\vec b$:\ \hrulefill\par\hrulefill\par\hrulefill
	    \end{enumerate}

    \item
	В базисе $\{\vec e_1, \vec e_2, \vec e_3\}$ заданы вектора $\vec a = \alpha_1 \vec e_1 + \alpha_2 \vec e_2 + \alpha_3 \vec e_3$ и $\vec b = \beta_1 \vec e_1 + \beta_2 \vec e_2 + \beta_3 \vec e_3$.
 
	    \begin{enumerate}
	        \item $(\mathbf a, \vec b)= $ \ \hrulefill\par\hrulefill
	        \item если $\{\vec e_1, \vec e_2, \vec e_3\}$ ОНБ, то $(\vec a, \vec b)=$ \hrulefill
	    \end{enumerate}
    
    \item Верно ли $\forall\ \alpha, \vec a, \vec b, \vec c$, что
    \begin{tasks}(2)
    \task $(\vec a, \vec b) = (\vec b, \vec a)$
    \task $[\vec a, \vec b] = [\vec b, \vec a]$
    \task	$\left[{\vec  {a}},[{\vec  {b}},{\vec  {c}}]\right]={\vec  {b}}\left({\vec  {a}}, {\vec  {c}}\right)-{\vec  {c}}\left({\vec  {a}}, {\vec  {b}}\right)$
    %\task $(\vec a, \alpha \vec b) = (\alpha \vec a, \vec b)$
    \task $[\vec a, \alpha \vec b] = -[\alpha \vec a, \vec b]$
    \end{tasks}

   	\item Упростите:
   	\begin{tasks}(1)
    \task $(\vec a-\vec b, \vec a+\vec b) = $
    \task $[\vec a-\vec b, \vec a+\vec b] = $
    \end{tasks}

    
        
    \item В правом ОНБ $\{\vec e_1, \vec e_2, \vec e_3\}$ заданы векторы $\vec a = \alpha_1 \vec e_1 + \alpha_2 \vec e_2 + \alpha_3 \vec e_3$ и\\ ${\vec b = \beta_1 \vec e_1 + \beta_2 \vec e_2 + \beta_3 \vec e_3}$. Найдите:
        \begin{tasks}(3)
    		\task $[\vec e_1, \vec e_1]=$ 
    		\task $[\vec e_1, \vec e_2]=$
    		\task $[\vec e_1, \vec e_3]=$
   		 	\task \begin{minipage}[c]{4em}
   		 		\vspace{3em}
   		 		$[\vec a, \vec b]=$
   		 		\vspace{3em}
   		 	\end{minipage}
   		\end{tasks}
   		
   	\item Найдите площадь треугольника, построенного на векторах $\vec a = (3\ 1\ 1)^T$ и $\vec b = (4\ 1\ 2)^T$ (координаты заданы в ОНБ).
   	\vspace{5em}
   		
   	\item Найти ортогональную проекцию вектора $\vec a = (1\ 2\ 3)^T$ на прямую с направляющим вектором $\vec b = (3\ 0\ 4)^T$ в ОНБ:


\end{enumerate}