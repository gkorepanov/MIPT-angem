\def \firstA {
    \item На эллипсе $\frac{x^2}{4} + y^2 = 1$ найти точки, из которых отрезок, соединяющий фокусы, виден под наибольшим углом. 
}

\def \firstB {
    	\item Составьте уравнения сторон квадрата, вписанного в эллипс $\frac{x^2}{a^2} + \frac{y^2}{b^2} = 1$, ${(a>b>0)}$. Какую часть площади, ограниченной эллипсом, составляет площадь этого квадрата?
}

\def \secondA {
    \item Составить уравнение эллипса, если точки $F_1(5, 1)$ и $F_2(-1, 1)$ являются фокусами, а прямая $x = \frac{31}{3}$ \ - одной из директрис
}

\def \secondB {
    \item Составить уравнение эллипса, если точка $F(-6, 2)$ является одним из фокусов, точка $A(2, 2)$ - концом большой оси, эксцентриситет равен $\frac{2}{3}$.
}

\def \secondC {
    \item Составить уравнение эллипса, если оси эллипса параллельны осям координат , точки $A(4, 0)$ и $B(0, 4)$ принадлежат эллипсу, а точка $B$ находится на расстоянии $3\sqrt{2}$ от одного из фокусов и на расстоянии $6$ от соответствующей директрисы.
}



\def \third {
    \item Пусть $O$ \  - цетр эллипса, $a$ и $b$ его полуоси, а $A$ и $B$ такие его точки, что прямые, содержащие $OA$ и $OB$, взаимно перпендикулярны.  
    \begin{enumerate}
        \item Доказать, что величина $\frac{1}{|OA|^2} + \frac{1}{|OB|^2}$ постоянна для всех возможных пар точек $A$ и $B$.
        \item Найти наибольшее и наименьшее значения длины отрезка $AB$
    \end{enumerate}
}


