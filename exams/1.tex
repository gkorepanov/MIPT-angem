\documentclass{angem}

\begin{document}
\sloppy
	\fancyhead[L]{18 октября 2018 г.}
	\fancyhead[R]{}
	\section*{Контрольная работа по аналитической геометрии №1}
	\begin{enumerate}
		\item В параллелограмме $ABCD$ точка $E$ лежит на диагонали $BD$, причем ${BE : ED = 1 : 2}$. Найти координаты точки плоскости в СК $\{A, \overrightarrow{AB}, \overrightarrow{AD}\}$, если известны ее координаты $x'$, $y'$ в СК $\{E, \overrightarrow{EC}, \overrightarrow{ED}\}$.
		
		\item Составить в ОДСК уравнения прямых, проходящих через точку $A(-1, 5)$ и равноудаленных от точек $B(3, 7)$, $C(1, -1)$.
		
		\item В ПДСК заданы точки $A(0, 0, 0), B(1, 2, 3), C(3, 2, 1)$ и $D(2, 0, 1)$, являющиеся вершинами тетраэдра. Найдите:
         \begin{enumerate}
         	\item объем тетраэдра
	        \item уравнение плоскости, содержащей основание $ABC$
	        \item высоту, проведенную к основанию $ABC$
	        \item уравнение прямой, перпендикулярной основанию $ABC$ и проходящей через $D$ (высота, опущенная из $D$ на $ABC$)
	    \end{enumerate}
	    
	    \item Точка $A$ лежит на прямой $\begin{cases}
	    	x-y-3=0\\2y+z=0\end{cases},$ расстояние от точки $A$ до прямой ${x=y=z}$ равно $\sqrt{6}$. Найти координаты точки $A$.
	    
	    \item 
	    Составить уравнение биссекторной плоскости того двугранного угла между плоскостями $x-z-5=0$ и $3x+5y+4z=0$, внутри которого лежит точка $A(1, 1, 1)$.
	\end{enumerate}
\end{document}